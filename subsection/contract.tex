Z definicji „Umowa jest to wzajemne uzgodnienie dwóch lub więcej stron mająca na celu obopólne dobro oraz określająca wzajemne obowiązki oraz prawa stron.” i 

Umowy zawierać można na wiele sposobów, każdy kraj dodaje swoje obostrzenia jednak w większości możliwe jest utworzenie umowy w wersji pisemnej z podpisem i egzemplarzem dla każdej ze stron.  

Każda umowa to odrębna sytuacja jednak wiele z nich podchodzi pod schemat, dzięki któremu możemy zautomatyzować czynności. Bazując na tym platforma umożliwia tworzenie takich oto schematów, aby mogły być w automatyczny sposób dostarczane do wielu klientów i przez wielu na raz wypełniane. 

Wypełnienie wiąże się z procesem z góry określonym przez który musi przejść użytkownik, aby uzyskać umową, którą będzie musiał jedynie wypełnić. Proces ten w większości przypadków to zebranie takich danych jakie potrzeba, aby idealnie przygotować umowę pod daną osobę. 

Umowy występują jednak pod różnymi poziomami skomplikowania. Część to proste jednostronne umowy z prostym systemem. Kolejne to umowy kilkunastu stronnicowe precyzyjnie określające wszystkie możliwe akcje. Wiąże się to z dużą ilością danych jakie trzeba zebrać, przetworzyć i w zgodzie z nimi zbudować umowę końcową. [7] 
