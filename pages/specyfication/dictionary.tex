\begin{description}
    \item[Szablon umowy (ang. Contract)] \hfill \\ Jest to encja opisująca precyzyjnie jak ma wyglądać formularz umowy dla użytkownika, jak ma wyglądać wygenerowana umowa końcowa. Opisuje wszystkie zależności oraz uwarunkowania. 

    \item[Część szablonu (ang. Contract part)] \hfill \\ Jest to fragment szablonu umowy, przechowywany do ponownego użycia. Najczęściej używany do elementów które powtarzają się w wielu umowach. 

    \item[Formularz umowy (ang. Contract form)] \hfill \\ Jest to formularz jaki musi wypełnić użytkownik, jeśli chce uzyskać wygenerowaną umowę. Wygląd i zawartość formularza definiują szablon umowy. 

    \item[Panel] \hfill \\ Cześć platformy odpowiedzialny za nawigację użytkownika po opcjach systemu. Panel pełni również opcję informacyjną, przedstawiając szereg najważniejszych danych dla użytkownika w zależności od posiadanych uprawnień. 

    \item[Uprawnienie (ang. Permission)] \hfill \\ Jest to encja precyzująca dostęp do danych akcji w systemie na przykład istnieje uprawnienie pod nazwą manage-users, które określa czy dany użytkownik może zarządzać użytkownikami w systemie. 

    \item[Konto (ang. Account)] \hfill \\ Jedno konto określa jednego użytkownika w systemie. Konto posiada określoną rolę i może wykonywać określone akcje. 

    \item[Rola (ang. Role)] \hfill \\ Jest to nazwany zbiór uprawnień w systemie. 

    \item[Użytkownik (ang. User)] \hfill \\ Podstawowy użytkownik, w systemie którego uprawnienia pozwalają jedynie na wypełnianie umów i używanie funkcji z tym związanych. 

    \item[Administrator] \hfill \\ Osoba posiadające pełne uprawnienia w systemie. 

    \item[Moduł] \hfill \\ Zbiór kodu po stronie backendowej jak i frontendowej rozszerzający podstawowe funkcjonalności systemu 

    \item[Generowanie umowy] \hfill \\ Zakolejkowany proces, podczas którego dane z formularza umowy użyte zostają w procesie budowy umowy na podstawie szablonu umowy. Finalnym efektem działania generatora jest plik PDF będącym gotową umową. Generator po wygenerowaniu umowy oddaję ją do modułu dostarczania. 

    \item[API (ang. Application Programming Interface)] \hfill \\ „Interfejs programowania aplikacji, czyli określony zestaw reguł, dzięki którym programy komputerowe mogą między sobą udostępniać dane – porozumiewać się ze sobą. Dzięki API możliwe jest korzystanie z funkcjonalności innych aplikacji w naszej aplikacji. API dostarcza specyfikacje podprogramów, struktur danych protokołów komunikacyjnych i klas obiektów.”ii 
\end{description}