Język ten rozszerza możliwości CSS o szereg funkcjonalności ułatwiających pracę ze stylami. Odpowiada ze opisanie wyglądu szablonu napisanego za pomocą HTML 

SCSS jest w pełni kompatybilny z css dzięki czemu możemy bez problemu zmienić rozszerzenia plików .css na .scss i skompilować taki kod. 

Kompilowaniem do .css zajmuje się tutaj SASS (ang. Syntactically Awesome Stylesheets) 