Język ten wykorzystywany jest jako fundament dla nowoczesnych frameworków frontendowych. Pozwala pisać dynamiczne strony internetowe. Jest to skryptowy język programowania. Ma możliwość działania po stronie klienta jak i serwera. Najczęściej jednak używany jest po stronie klienta w przeglądarkach internetowych.  

Dzięki kompilatorom takim jak Webpack, Babel możliwe jest pisanie w najnowszej wersji, standardzie ECMAScript. Kod następnie kompilowany jest to starszych wersji, aby mógł działać na większości dzisiejszych przeglądarek. Niestety przeglądarki wciąż pozostają w tyle z implementacją nowych funkcjonalności i takie działanie jest nieuniknione, jeśli chcemy zachowywać standardy i funkcjonalność. 

Najnowsza wersja to ECMAScript 2018, przeglądarki natomiast zatrzymały się na obsłudze wersji z przed 2 lat i to jedynie częściowo. 