Docker to narzędzie zaprojektowane w celu ułatwienia tworzenia, wdrażania i uruchamiania aplikacji przy użyciu kontenerów. Kontenery pozwalają programiście na spakowanie aplikacji ze wszystkimi częściami, których potrzebuje, takimi jak biblioteki i inne zależności, i wysłanie jej w jednym pakiecie. W ten sposób, dzięki kontenerowi, deweloper może mieć pewność, że aplikacja będzie działać na każdej innej maszynie, bez względu na wszelkie niestandardowe ustawienia, które maszyna może mieć, a które mogą się różnić od tych, które są używane do pisania i testowania kodu, a także od serwera produkcyjnego.

W pewnym sensie, Docker jest trochę jak wirtualna maszyna. Ale w przeciwieństwie do maszyny wirtualnej, zamiast tworzyć cały wirtualny system operacyjny, Docker pozwala aplikacjom korzystać z tego samego jądra Linuksa co system, na którym są uruchomione i wymaga tylko dostarczenia aplikacji z rzeczami, które nie są jeszcze uruchomione na komputerze hosta. Daje to znaczący wzrost wydajności i zmniejsza rozmiar aplikacji.