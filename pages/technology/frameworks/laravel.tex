Jest to najpopularniejszy framework do tworzenia aplikacji internetowych w języku PHP. Zostawił w tyle większość konkurencji, umożliwiając bezkompromisową pracę.  
Jest zbudowany na fundamencie wzorca MVC (ang. Model View Controler).  
Słynie z efektywnej i bardzo eleganckiej składni. Jedną z głównych cech jest czystość kodu i struktury jaką oddaje nam do użytku Laravel.
Pozostałe cechy które warto wymienić to: 

\begin{itemize}
    \item Prosty i szybki system routingu,
    \item gotowy kontener wstrzykiwania zależności z ogromem funkcji dodatkowych i konfiguracyjnych,
    \item intuicyjny system ORM dla praktycznie wszystkich najnowszych baz danych,
    \item precyzyjny system migracji do bazy danych,
    \item potężny system do obsługi wypełniania bazy danych,
    \item zadania w tle,
    \item obsługa i możliwość tworzenia komend,
    \item mechanizm kolejkowania zadań,
    \item system mailingowy,
    \item system notyfikacji,
    \item implementacja wielu wzorców projektowych do konkretnych zadań,
    \item system eventowy,
    \item broadcasting
  \end{itemize}