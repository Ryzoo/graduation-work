Jest to najpopularniejszy framework do tworzenia aplikacji internetowych w języku PHP. Zostawił w tyle większość konkurencji, umożliwiając bezkompromisową pracę.  
Jest zbudowany na fundamencie wzorca MVC (ang. Model View Controler).  
Słynie z efektywnej i bardzo eleganckiej składni. Jedną z głównych cech jest czystość kodu i struktury jaką oddaje nam do użytku Laravel.
Pozostałe cechy które warto wymienić to: 

\begin{itemize}
    \item Prosty i szybki system routingu 
    \item Gotowy kontener wstrzykiwania zależności z ogromem funkcji dodatkowych i konfiguracyjnych 
    \item Intuicyjny system ORM dla praktycznie wszystkich najnowszych baz danych 
    \item Precyzyjny system migracji do bazy danych 
    \item Potężny system do obsługi wypełniania bazy danych  
    \item Zadania w tle  
    \item Obsługa i możliwość tworzenia komend 
    \item Mechanizm kolejkowania zadań 
    \item System mailingowy  
    \item System notyfikacji   
    \item Implementacja wielu wzorców projektowych do konkretnych zadań 
    \item System eventowy 
    \item Broadcasting
  \end{itemize}